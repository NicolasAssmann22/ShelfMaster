\documentclass{article}
\usepackage[a4paper, left=3cm, right=3cm, top=4cm]{geometry}

\title{ShelfMaster Projektanforderungen}
\begin{document}

\maketitle

\section{Problemstellung und Lösung}
  \subsection{Problem}
  Im Alltag wird häufig der Überblick über die in Schränken oder Regalen gelagerten Gegenstände
  verloren. Nutzende haben Schwierigkeiten, bestimmte Gegenstände schnell zu finden, besonders wenn es an
  einer konsistenten Organisation fehlt.

  \subsection{Lösung}
  Die \textit{ShelfMaster}-Anwendung hift den Nutzenden dabei, ihre Schränke, Regale und Gegenstände
  systematisch zu verwalten und über eine intuitive Suchfunktion schnell auf bestimmte Objekte zuzugreifen.
  Mit hierarchischer Organisation (Schränke, Fächer, Kategorien) und der Suchfunktion wird der Zugriff
  auf Gegenstände erheblich vereinfacht.

\section{Erweiterte Funktionen}
  \begin{itemize}
    \item \textbf{Mehrbenutzerunterstützung}: Bietet die Möglichkeit, Schränke mit anderen zu teilen, damit mehrere Nutzer sie verwalten können.
    \item \textbf{Inventarhistorie}: Speicherung der Veränderungshistorie eines Gegenstands für eine bessere Nachverfolgbarkeit.
    \item \textbf{Lagerungsvorschlag}: Schlägt aufgrund des Objektes einen sinnvollen Ort zur Lagerung vor.
    \item \textbf{Visuelle Organisation mit Fotos}: Nutzende könnten Fotos der Fächer und Gegenstände hinzufügen, um eine visuelle Übersicht zu bekommen.
    \item \textbf{Erweiterte Statistik}: Zeigt Nutzungsstatistiken wie häufig genutzte Gegenstände, selten genutzte Fächer oder Gegenstandsrotation.
  \end{itemize}
\end{document}